\subsection{Spin Estimates}\label{sec:spin-magnitudes}
We now look at the constraints placed on the spin of the slowly spinning simulated BNS sources.  Even though the simulations occupy a vary small fraction of the prior volume, most posterior distributions span the majority of the prior range for spin magnitudes. For non-precessing systems with a relatively stationary orbital plane with respect to the line-of-sight, varying the spin of the compact objects has a similar effect on the phase evolution of the GW as varying the mass ratio, resulting in a strong degeneracy between the two parameters.  Modulation of the GWs from precession of the orbital plane can break this degeneracy \citep{Vecchio_2004,Lang_2006,Vitale_2014,Chatziioannou_2014}, however only systems with high spins that are misaligned with the orbital angular momentum significantly precess. Non-precessing systems, with either slow or aligned spins, only provide phase information and suffer the most from this degeneracy.  The simulated sources in this study fall in the latter category.  Figure \ref{fig:spinPDF} shows the average PDF for the spin of the most and least massive components, $\chi_1$ and $\chi_2$, respectively.

The spin of the more massive component has a larger effect on the GW, and is therefore systematically better constrained, as seen in Fig. \ref{fig:spinPDF}.  For both spins, however, the posterior shows slow spins to be only slightly more probable than high spins for most sources.
  
  
  