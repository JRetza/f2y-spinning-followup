We now look at the constraints we can place on slowly spinning BNS sources.  Typically the modulation present in GWs from sytems with high spin and misalignment between the spin and orbital angular momenta allow for strong constraints to be made on spin parameters, breaking much of the degeneracy with mass ratio (CITE).  Non-precessing systems, with either low or aligned spins, have strong degeneracies between spin and mass ratio.  The simulated sources in the study fall in the latter catagory, which is largely why the mass constraints in the previous section are so weak.  Likewise, constraints placed on the spin magnitudes are also weak.  Figure \ref{fig:spinPDF} shows the average PDF for the spin of the most and least massive compenents: $\chi_1$ and $\chi_2$, respectively.

The spin of the more massive component has a larger effect on the GW, and is therefore systematically better constrained, as seen in Fig. \ref{fig:spinPDF}.  For both spins, however, the posterior shows low spins to be only slightly more probable than high spins, on average.