\subsection{Spin Estimates}\label{sec:spin-magnitudes}
We now look at the constraints placed on the spin of the slowly spinning simulated BNS sources.  Even though the simulations occupy a small fraction of the spin-magnitude prior volume, most posterior distributions span the majority of the prior range. For non-precessing systems, where the orbital plane is stationary with respect to the line-of-sight, varying the spin of the compact objects has a similar effect on the phase evolution of the GW as varying the mass ratio. This results in a strong degeneracy between the two parameters.  Modulation of the GWs from precession of the orbital plane can break this degeneracy \citep{Vecchio_2004,Lang_2006,Vitale_2014,Chatziioannou_2014}; however, only systems with large spins that are misaligned with the orbital angular momentum significantly precess. Non-precessing systems, with either low or aligned spins, suffer the most from this degeneracy as the only information regarding the mass and spin is encoded in the phase of the GW.  The simulated sources in this study fall in the latter category of low spins.  Figure  \ref{fig:spinPDFcred} shows the distribution of Gaussian kernel density estimates of the PDFs for the spin of the most and least massive components, $\chi_1$ and $\chi_2$, respectively.  The labeled regions of figure \ref{fig:spinPDFcred} bound the specified percent of PDFs as a function of spin, where the $90\%$ region, for example, is bounded by the $5$th and $95$th percentiles of the PDFs at each spin value. 

The spin of the more massive component has a larger effect on the GW, and is therefore systematically better constrained, as seen in Figure \ref{fig:spinPDFcred}.  For both spins, however, the posterior shows slow spins to be only slightly more probable than high spins for most sources. The mean (median) $50\%$ upper limits on $\chi_1$ and $\chi_2$ are $0.319$ ($0.302$) and $0.424$ ($0.419$) respectively; the $90\%$ upper limits are $0.707$ ($0.699$) and $0.855$ ($0.859$).
  
  
  
  
  
  
  
  
  
  
  
  
  