\section{Source location}\label{sec:extrinsic}

Having discussed how GW observations can measure the intrinsic properties of their source systems, we now consider the measurement of extrinsic parameters, specifically the sky position (section \ref{sec:sky}) and the distance (section \ref{sec:distance}). On their own, these are not useful for understanding the physics of compact objects, but they are central to the success of multimessenger astronomy. The sky position is required in order to direct telescopes for electromagnetic follow-up and to verify that any observed transients do coincide with the source of the gravitational waves. The distance also aids electromagnetic follow-up as it allows cross-reference with galaxy catalogues to find the most probably source locations \citep{Nissanke_2012,Fan_2014}. Even without an observed counterpart, the posterior for the (three-dimensional) position allows us to assign a probability that the source resided in given galaxies; combined the redshift of these galaxies (measured electromagnetically) with the gravitational-wave luminosity distance gives a measure of the Hubble constant free of the usual systematics \citep{Schutz_1986,Del_Pozzo_2012}. For our population of slowly spinning neutron stars, we do not expect measurement of the extrinsic parameters to be affected by the inclusion of spin in the analysis.

\subsection{Sky localization}\label{sec:sky}

We characterize sky localization using credible regions, the smallest sky area that encompasses a given total posterior probability. The credible region for a total posterior probability $p$ is defined as
\begin{equation}
\mathrm{CR}_p = \underset{A}{\arg\!\max} \int_A \mathrm{d}\boldsymbol{\Omega} P_{\Omega}(\boldsymbol{\Omega}),
\label{eq:CR}
\end{equation}
where $P_{\Omega}(\boldsymbol{\Omega})$ is the posterior PDF over sky position $\boldsymbol{\Omega}$, and $A$ is the sky area integrated over \citep{Sidery_2014}. We also consider the searched area $A_\ast$, the area of the smallest credible region that includes the true location.

For electromagnetic (EM) observatories to be able to conduct follow-up of a GW detection, they need an accurate sky location. This must be provided promptly, while there is still a visible transient. The fully spinning PE is computationally expensive and so slow to compute. There are alternative methods that can provide sky localization more quickly. The most expedient is \textsc{bayestar}, this uses output from the detection pipeline to rapidly compute sky position \citep{Singer_2014}. \textsc{bayestar} can compute sky positions with a latency of $\sim30~\mathrm{s}$ \citep{Berry_2014}. Between the low-latency \textsc{bayestar} and the high-latency fully spinning PE, there is the medium-latency option of performing non-spinning PE with TaylorF2 waveforms. This requires $\sim10^5~\mathrm{s}$ of wall time to complete PE, with the exact time depending upon the degree of parallelization. Despite only using information form the detection triggers, rather than full waveforms, it has been shown that \textsc{bayestar} produces sky areas fully consistent with non-spinning PE results, provided that there was a trigger from all detectors in the network \citep{Singer_2014,Berry_2014}. Having now performed a full spinning analysis, we can compare the results of high-latency PE with the more expedient methods of inferring sky position.

In Figure~\ref{fig:sky} we show the cumulative distributions of recovered $50\%$ credible regions, $90\%$ credible regions and searched areas. All three quantities show good agreement across all PE techniques.\footnote{Performing a Kolmogorov--Smirnov test gives some number, which will be added later.} Including spin in the analysis does not change the average ability to locate the source on the sky.
