\section{Sky localization}

Having discussed how GW observations can measure the intrinsic properties of their source systems, we now consider the measurement of extrinsic parameters such as source location. On their own, these are not useful for understanding the physics of compact objects, but they are central to the success of multimessenger astronomy.

We characterize sky localization using credible regions, the smallest sky area that encompasses a given total posterior probability. The credible region for a total posterior probability $p$ is defined as
\begin{equation}
\mathrm{CR}_p = \underset{A}{\arg\!\max} \int_A \mathrm{d}\boldsymbol{\Omega} P_{\Omega}(\boldsymbol{\Omega}),
\label{eq:CR}
\end{equation}
where $P_{\Omega}(\boldsymbol{\Omega})$ is the posterior PDF over sky position $\boldsymbol{\Omega}$, and $A$ is the sky area integrated over \citep{Sidery_2014}. We also consider the searched area $A_\ast$, the area of the smallest credible region that includes the true location.

For electromagnetic observatories to be able to conduct follow-up of a GW detection, they need an accurate sky location. This must be provided promptly, while there is still a visible transient. The fully spinning PE is computationally expensive and so slow to compute. There are alternative methods that can provide sky localization more quickly. The most expedient is \textsc{bayestar}, this uses output from the detection pipeline to rapidly compute sky position \citep{Singer_2014}. \textsc{bayestar} can compute sky positions with a latency of $\sim30~\mathrm{s}$ \citep{Berry_2014}. Between the low-latency \textsc{bayestar} and the high-latency fully spinning PE, there is the medium-latency option of performing non-spinning PE with TaylorF2 waveforms. This requires $\sim10^5~\mathrm{s}$ of wall time to complete PE, with the exact time depending upon the degree of parallelization. Despite only using information form the detection triggers, rather than full waveforms, it has been shown that \textsc{bayestar} produces sky areas fully consistent with non-spinning PE results, provided that there was a trigger from all detectors in the network \citep{Singer_2014,Berry_2014}. Having now performed a full spinning analysis, we can compare the results of high-latency PE with the more expedient methods of inferring sky position.

