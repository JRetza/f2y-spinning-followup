\section{Source Location}\label{sec:extrinsic}

Having discussed how GW observations can measure the intrinsic properties of their source systems, we now consider the measurement of extrinsic parameters, specifically the sky position (section \ref{sec:sky}) and the distance (section \ref{sec:distance}). These are central to the success of multimessenger astronomy. The sky position is required in order to direct telescopes for electromagnetic (EM) follow-up and to verify that any observed transients do coincide with the source of the gravitational waves. The distance also aids electromagnetic follow-up as it allows cross-reference with galaxy catalogs to find the most probable source locations \citep{Nissanke_2012,Hanna:2013,Fan_2014,Blackburn:2014rqa}. Even without an observed counterpart, the posterior for the (three-dimensional) position allows us to assign a probability that the source resides in given galaxies; combining the redshift of these galaxies (measured electromagnetically) with the gravitational-wave luminosity distance gives a measure of the Hubble constant free of the usual systematics \citep{Schutz_1986,Del_Pozzo_2012}. For our population of slowly spinning NSs, we do not expect the measurement of the extrinsic parameters to be affected by the inclusion of spin in the analysis.

\subsection{Sky Localization}\label{sec:sky}

In order for EM observatories to follow-up a GW detection, they need an accurate sky location. This must be provided promptly, while there is still a visible transient. Parameter estimation while accounting for spin is computationally expensive and slow to complete (see appendix \ref{ap:CPU}). There are alternative methods that can provide sky localization more quickly. The most expedient is \textsc{bayestar}, which uses output from the detection pipeline to rapidly compute sky position \citep{Singer:2015ema}. \textsc{bayestar} can compute sky positions with a latency of a few seconds. Between the low-latency \textsc{bayestar} and the high-latency full parameter estimation, there is the medium-latency option of performing non-spinning parameter estimation with computationally cheap TaylorF2 waveforms. This requires hours of wall time to complete analyses, with the exact time depending upon the degree of parallelization. Despite only using information from the detection triggers, rather than full waveforms, it has been shown that \textsc{bayestar} produces sky areas for BNS signals fully consistent with non-spinning parameter estimation results, provided that there was a trigger from all detectors in the network \citep{Singer_2014,Berry_2014,Singer:2015ema}. Having now performed a full spinning analysis, we can compare the results of high-latency parameter estimation with the more expedient methods of inferring sky position.

In Figure~\ref{fig:sky} we show the cumulative distributions of recovered $50\%$ credible regions, $90\%$ credible regions, and searched areas. All three quantities show good agreement across all parameter-estimation techniques. %\footnote{Performing a Kolmogorov--Smirnov test gives some number, which will be added later.}
For the for the slowly-spinning BNSs considered here, including spin in the analysis does not change the average ability to localize sources on the sky.

  
  
  
  
  
  
  
  
  
  
  
  
  
  
  