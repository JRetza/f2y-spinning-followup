We can consider sky localization in greater detail by comparing areas on an event-by-event basis and not just the cumulative distribution across the population. Doing this, we confirm that sky localization is consistent between approaches for any given event. We use the medium-latency, non-spinning TaylorF2 analysis as a reference point, and compare the ratio of sky areas. To summarize the variation in sky areas computed in different analyses, we use the log ratio
\begin{equation}
\mathcal{R}_A^X = \log_{10}\left(\frac{A^X}{A^\mathrm{NS}}\right),
\end{equation}
where $A^X$ is a credible region or the searched area as determined by method $X$ and $A^\mathrm{NS}$ is the same quantity from the non-spinning analysis. The log ratio $\mathcal{R}_A^X$ is zero when analysis $X$ agrees with the non-spinning results. Considering all $250$ events, the mean and standard deviation of the log ratio is given in Table~\ref{tab:sky-ratio}. For the purposes of EM follow-up, there is no significant difference between analyses.\footnote{The non-spinning analysis was performed with \textsc{LALInference\_nest} while the spinning analysis was performed with \textsc{LALInference\_MCMC} \citep{Veitch_2014}; therefore the consistency between analyses additionally shows the consistency of results from different sampling algorithms.} The computationally expensive fully spinning analysis does not improve sky localization: there is no disadvantage in using the lower-latency results for EM follow-up of slowly spinning BNSs.
  
  
  
  
  
  
  
  