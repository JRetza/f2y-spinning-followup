\subsection{Mass Estimates}\label{sec:mass}
To maximize sampling efficiency, model parameterizations are chosen to minimize degeneracies between parameters.  To leading order, the post-Newtonian expansion of the waveform's phase evolution depends on the \emph{chirp mass}, $\mathcal{M}_\mathrm{c} = (m_1 m_2)^{3/5} (m_1 + m_2)^{-1/5}$, making it a \emph{very} well constrained parameterization of binary mass.  The second mass parameter used is the mass ratio $q = m_2/m_1$, where $0 < q \leq 1$.  Detectors are much less sensitive to the mass ratio, and strong degeneracies with spin make constraints on $q$ even worse \citep{Cutler_1994}.  It is primarily the uncertainty in $q$ that governs the uncertainty in component masses $m_1$ and $m_2$.

Figure \ref{fig:mass_pdfs} shows the superimposed, one-dimensional marginal posterior PDFs and cumulative density functions (CDFs) for the chirp mass (centered on each mean) and mass ratio for all $250$ events.  Also shown are the PDFs and CDFs averaged over all $250$ posteriors, representing a typical event's posterior distributions.  Chirp-mass distributions are usually well approximated by normal distributions about the mean, while mass ratio estimates have broad support across most of the prior range.

To trace individual parameter uncertainties across the population we use the fractional uncertainties $\sigma_{\{\mathcal{M}_\mathrm{c},~q\}}/\{\langle\mathcal{M}_\mathrm{c}\rangle,~\langle q\rangle\}$, where $\sigma_x$ and $\langle x\rangle$ are the standard deviation and mean of the distributions respectively.  Figure \ref{fig:mass_std_snr} shows the distribution of chirp-mass and mass-ratio fractional uncertainties, with colors corresponding to the S/N recovered by the detection pipeline. The average fractional uncertainties in chirp mass and mass ratio for the simulated population are $0.0675\%$ and $28.7\%$.

The fractional uncertainties for both the chirp mass and the mass ratio both decrease as S/N increases as shown in \ref{fig:Mc_q_std_snr}, which also shows results from the non-spinning analysis. All except the $\sigma_q/\langle q\rangle$ from the spinning analysis appear to be inversely proportional to the S/N: this is better fit as $\propto \rho^{-1/2}$. The mass-ratio uncertainty from the spinning analysis does not improve as rapidly with increasing S/N as a consequence of the mass--spin degeneracy.
  
Projecting the very tightly constrained chirp mass and poorly constrained mass ratio $90\%$ credible region from $\mathcal{M}_\mathrm{c}$--$q$ space into component-mass space makes it obvious how important mass ratio uncertainties are for extracting astrophyical information.  The credible regions in component-mass space are very narrow bananas that lie along lines of constant chirp mass, bounded by the constraints on mass ratio (see Figure \ref{fig:comp_masses} for some examples posteriors).
  
  
  
  
  
  
  
  