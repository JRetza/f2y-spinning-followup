\subsection{Mass Estimates}

For the sake of sampling efficiency, it is common to reparameterize the model to reduce the degeneracy between parameters, particularly those specifing the binary's masses.  GW detectors are most sensitive to a combination of component masses referred to as the chirp mass, $\mathcal{M}_\mathrm{c} = (m_1 m_2)^{3/5} (m_1 + m_2)^{-1/5}$.  For this study, we use the assymetric mass ratio $q = m_2/m_1$, where $0 < q < 1$, as the second mass parameter.  Detectors are much less sensitive to the mass ratio, and strong degeneracies with spin make constraints on $q$ even worse.  It is primarily the uncertainty in $q$ that governs the uncertainty in component masses $m_1$ and $m_2$.

Figure \ref{fig:mass_std} shows the distribution of mass standard deviations for the 250 simulated signals from the three analyses previously described.  The consistency between the non-spinning analyses shows that the drastic increases in the uncertainty of mass parameters from the spinning analysis is due purely to degeneracies with spin, and not systematic difference between waveform families.