\subsection{Mass Estimates}\label{sec:mass}
To maximize sampling efficiency, model parameterizations are chosen to minimize degeneracies between parameters.  To leading order, the post-Newtonian expansion of the waveform's phase evolution depends on the \emph{chirp mass}, $\mathcal{M}_\mathrm{c} = (m_1 m_2)^{3/5} (m_1 + m_2)^{-1/5}$, making it a \emph{very} well constrained parameterization of binary mass.  The second mass parameter used is the mass ratio $q = m_2/m_1$, where $0 < q \leq 1$.  Detectors are much less sensitive to the mass ratio, and strong degeneracies with spin make constraints on $q$ even worse \citep{Cutler_1994}.  It is primarily the uncertainty in $q$ that governs the uncertainty in component masses $m_1$ and $m_2$.

Figure \ref{fig:mass_pdfs} shows the superimposed, one-dimensional marginal posterior PDFs and cumulative density functions (CDFs) for the chirp mass (centered on each mean) and mass ratio for all $250$ events.  As a representation of a typical event's posterior distribution, we show the average PDFs and CDFs, where the average is taken over all $250$ posterior PDFs and CDFs at each point. Chirp-mass distributions are usually well approximated by normal distributions about the mean, while mass ratio estimates have broad support across most of the prior range, whereas the simulated population had a narrower range between $0.75$ and $1$.

To trace individual parameter uncertainties across the population we use the fractional uncertainties in chirp mass $\sigma_{\mathcal{M}_\mathrm{c}}/\langle\mathcal{M}_\mathrm{c}\rangle$ and mass ratio $\sigma_q/\langle q\rangle$. The chirp mass and mass ratio conveniently cover mass space (which is why they are used for sampling), but the total mass $M = m_1 + m_2$ is also of interest for determining the end product of the merger, so we also plot the fractional uncertainty $\sigma_M/\langle M\rangle$. The mean (median) fractional uncertainties in chirp mass, mass ratio and total mass for the simulated population are $0.0675\%$ ($0.0640\%$), $28.7\%$ ($28.4\%$) and $6.15\%$ ($5.81\%$) respectively.

The fractional uncertainties for the chirp mass, mass ratio and total mass all decrease as S/N increases, as shown in Figure \ref{fig:Mc_q_std_snr}, which also shows results from the non-spinning analysis. As expected from Fisher-matrix studies \citep[e.g.,][]{FinnChernoff}, most appear to be inversely proportional to the S/N: the exception is $\sigma_q/\langle q\rangle$ from the spinning analysis, which better fit as $\propto \rho_\mathrm{net}^{-1/2}$. We do not suspect there is anything fundamental about the $\propto \rho_\mathrm{net}^{-1/2}$, rather it is a useful rule-of-thumb. The behaviour can still be understood from a Fisher-matrix perspective, which predicts a Gaussian probability distribution (with width $\propto \rho_\mathrm{net}^{-1}$). Since the mass ratio is constrained to be $0 \leq q \leq 1$, if the width of a Gaussian is large, it is indistinguishable from a uniform distribution and the standard deviation tends to a constant $1/\sqrt{12} \simeq 0.289$. When the width of the Gaussian is small ($\lesssim 0.1$), the truncation of the distribution is negligible and the standard deviation behaves as expected, as is the case for the non-spinning results. The standard deviations obtained for the spinning runs lie in the intermediate regime, between being independent of S/N and scaling inversely with it; the mean (median) standard deviation $\sigma_q$ is $0.182$ ($0.183$).\footnote{The uncertainty  for the symmetric mass ratio $\eta = m_1m_2/(m_1 + m_2)^2$, which is constrained to be $0 \leq \eta \leq 1/4$, does scale approximately as $\rho_\mathrm{net}^{-1}$. The mean (median) standard deviation $\sigma_\eta$ for the spinning runs is $2.00\times 10^{-2}$ ($1.95\times 10^{-2}$).} The mass--spin degeneracy broadens the posteriors for both the chirp mass, the mass ratio and the total mass; a consequence of the broadening for the mass ratio is that the uncertainty does not decrease as rapidly with S/N (over the range considered here). We further examine the impact of spin on mass measurements in section \ref{subsec:prior_constraints}.
  
Projecting the tightly constrained chirp mass and poorly constrained mass ratio $90\%$ credible region from $\mathcal{M}_\mathrm{c}$--$q$ space into component-mass space makes it obvious how important mass-ratio uncertainties are for extracting astrophysical information.  The credible regions in component-mass space are narrow bananas that lie along lines of constant chirp mass, bounded by the constraints on mass ratio (see Figure \ref{fig:comp_masses} for some examples posteriors).
  
  
  
  
  