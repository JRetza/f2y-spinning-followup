\subsection{Mass Estimates}

For the sake of sampling efficiency, it is common to reparameterize the model to reduce the degeneracy between parameters, particularly those specifing the binary's masses.  GW detectors are most sensitive to a combination of component masses referred to as the chirp mass, $\mathcal{M}_\mathrm{c} = (m_1 m_2)^{3/5} (m_1 + m_2)^{-1/5}$.  For this study, we use the assymetric mass ratio $q = m_2/m_1$, where $0 < q < 1$, as the second mass parameter.  Detectors are much less sensitive to the mass ratio, and strong degeneracies with spin make constraints on $q$ even worse.  It is primarily the uncertainty in $q$ that governs the uncertainty in component masses $m_1$ and $m_2$.

Figure \ref{fig:mass_pdfs} shows the superimposed, marginalized one-dimensional posterior probability density functions (PDFs) and cumulative density functions (CDFs) for the chirp-mass (centered on each mean) and mass ratio for all 250 events, along with the average distribution.  Both distributions are unimodal,

Figure \ref{fig:mass_std_snr} shows the distribution of chirp mass and mass ratio uncertainties (standard devations), with colors corresponding to the S/N measured by the detection pipeline.  Chirp mass is recovered with a standard deviation of $8.6 \times 10^{-4} \mathrm{M}_\odot$ and mass ratio $0.18$, on average, with both generally increasing as S/N decreases.