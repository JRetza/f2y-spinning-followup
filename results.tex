\subsection{Mass Estimates}

For the sake of sampling efficiency, it is common to reparameterize the model to reduce the degeneracy between parameters, particularly those specifying the binary's masses.  All analyses assume uniform priors in component masses between $0.6~\mathrm{M}_\odot$ and $5.0~\mathrm{M}_\odot$.  GW detectors are most sensitive to the \emph{chirp mass} $\mathcal{M}_\mathrm{c} = (m_1 m_2)^{3/5} (m_1 + m_2)^{-1/5}$.  We use the asymmetric mass ratio $q = m_2/m_1$, where $0 < q \leq 1$, as the second mass parameter.  Detectors are much less sensitive to the mass ratio, and strong degeneracies with spin make constraints on $q$ even worse \citep{Cutler_1994}.  It is primarily the uncertainty in $q$ that governs the uncertainty in component masses $m_1$ and $m_2$.

Figure \ref{fig:mass_pdfs} shows the superimposed, one-dimensional marginal posterior PDFs and cumulative density functions (CDFs) for the chirp mass (centered on each mean) and mass ratio for all $250$ events.  Also shown are the PDFs and CDFs averaged over all $250$ posteriors. Chirp-mass estimates are typically pseudo-normally distributed about the mean, while mass ratio estimates have support across most of the prior range.

From these one-dimensional PDFs we extract the standard deviations as point estimates of parameter uncertainties.  Figure \ref{fig:mass_std_snr} shows the distribution of chirp-mass and mass-ratio uncertainties, with colors corresponding to the S/N measured by the detection pipeline. On average, chirp mass is recovered with a standard deviation of $8.6 \times 10^{-4} \mathrm{M}_\odot$ and mass ratio $0.18$, with both generally increasing as S/N decreases.