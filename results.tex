\subsection{Mass Estimates}

For the sake of sampling efficiency, it is common to reparameterize the model to reduce the degeneracy between parameters, particularly those specifing the binary's masses.  We use the convention that $m_1$ is larger component mass and $m_2$ is the smaller.  All analyses assume uniform priors in component masses between $0.6~\mathrm{M}_\odot$ and $5.0~\mathrm{M}_\odot$.  GW detectors are most sensitive to a combination of component masses referred to as the chirp mass, $\mathcal{M}_\mathrm{c} = (m_1 m_2)^{3/5} (m_1 + m_2)^{-1/5}$.  For this study, we use the assymetric mass ratio $q = m_2/m_1$, where $0 < q < 1$, as the second mass parameter.  Detectors are much less sensitive to the mass ratio, and strong degeneracies with spin make constraints on $q$ even worse \citep{Cutler_1994}.  It is primarily the uncertainty in $q$ that governs the uncertainty in component masses $m_1$ and $m_2$.

Figure \ref{fig:mass_pdfs} shows the superimposed, marginalized one-dimensional posterior PDFs and cumulative density functions (CDFs) for the chirp mass (centered on each mean) and mass ratio for all $250$ events.  Also shown are the PDFs and CDFs averaged over all $250$ posteriors, as well as the distributions of simulated sources.  Both distributions are unimodal, with the mass-ratio distributions often extending to the equal-mass limit.  The average chirp-mass posterior PDF is consistent with the simulated distribution.  However, the mass-ratio PDFs show support across the majority of the prior, extending beyond the range of the simulated distribution.  The resulting average distribution is much broader than the simulated distribution.

From these one-dimensional PDFs we extract the standard deviations as point estimates of the parameter uncertainties.  Figure \ref{fig:mass_std_snr} shows the distribution of chirp-mass and mass-ratio uncertainties, with colors corresponding to the S/N measured by the detection pipeline. On average, chirp mass is recovered with a standard deviation of $8.6 \times 10^{-4} \mathrm{M}_\odot$ and mass ratio $0.18$, with both generally increasing as S/N decreases.