\section{Spinning Analysis}
\label{sec:spin}

\citet{Singer_2014} details the detection, low-latency localization, and medium-latency (i.e.\ non-spinning) follow-up of the simulated signals in 2015. In this work we perform the expensive task of full parameter estimation that accounts for non-zero compact-object spin. Whereas \citet{Singer_2014} used the (non-spinning) TaylorF2 waveform model, we make use of the SpinTaylorT4 waveform model \citep{Buonanno_2003,Buonanno_2009}, parameterized by the fifteen parameters that uniquely define a circularized compact binary inspiral.\footnote{The fifteen parameters are two masses (either component masses or the chirp mass and mass ratio); six spin parameters describing the two spins (magnitudes and orientations); two coordinates for sky position; distance; an inclination angle; a polarization angle; a reference time, and the orbital phase at this time \citep[see][for more details]{Veitch_2014}. The masses and spins are intrinsic parameters which control the evolution of the binary, the others are extrinsic parameters which describe its orientation and position.}

We assume the objects to be point masses with no tidal interactions.  The estimation of tidal parameters using post-Newtonian approximations is rife with systematic uncertainties that are comparable in magnitude to statistical uncertainties \citep{Yagi_2014,Wade_2014}. Though marginalizing over uncertainties in tidal parameters can affect estimates of other parameters, the fact that tidal interactions only impact the evolution of the binary at late times (only having a measurable impact at frequencies above $\sim450~\mathrm{Hz}$; \citealt{Hinderer_2010}) limits both their measurability and the resulting biases in other parameter estimates caused by ignoring them \cite{Damour_2012}.

The simulated population of BNS systems contains slowly spinning NSs, with masses between $1.2~\mathrm{M}_\odot$ and $1.6~\mathrm{M}_\odot$ and spin magnitudes $\chi < 0.05$.  This choice was motivated by the characteristics of NSs found thus far in Galactic BNS systems expected to merge within a Hubble time through GW emission. However, NSs \textit{outside} of BNS systems have been observed with spins as high as $\chi = 0.4$ \citep{Hessels_2006,Brown_2012}, and depending on the NS equation of state (EOS) could theoretically have spins as high as $\chi \lesssim 0.7$ \citep{Lo_2011} without breaking up.  For these reasons, the prior assumptions used for Bayesian inference of source parameters are more broad than the spin range of the simulated source population.

To simulate a real analysis scenario where the class of compact binary and the NS EOS are not known, we use uniform priors in component masses between $0.6~\mathrm{M}_\odot$ and $5.0~\mathrm{M}_\odot$ to avoid any prior constraints on mass posteriors, and our standard BH spin prior: uniform in spin magnitudes $\chi_{1,\,2} \sim U(0, 1)$ and isotropic in spin orientation. Prior distributions for the location and orientation of the binary match that of the simulated population, i.e.\ isotropically oriented and uniform in volume (out to a maximum distance of $218.9~\mathrm{Mpc}$, safely outside the detection horizon, which is $\sim137~\mathrm{Mpc}$ for a $1.6~\mathrm{M}_\odot$--$1.6~\mathrm{M}_\odot$ binary).\footnote{The mean (median) true distance for the set of $250$ events is $52.1~\mathrm{Mpc}$ ($47.8~\mathrm{Mpc}$), and the maximum is $124.8~\mathrm{Mpc}$.}  Choosing any particular upper bound for spin magnitude would require either assuming hard constraints on NS spin-up, which are based upon observations with hard-to-quantify selection effects, or making assumptions regarding the unknown EOS of NSs. For these reasons we choose not to rule out compact objects with high spin a priori by using an upper limit of $\chi < 1$, encompassing all allowed NS and BH spins.  In section \ref{subsec:prior_constraints} we look at more constraining spin priors, and particularly how such choices can affect mass estimates.

We describe parameter-estimation accuracy using several different quantities, depending upon the parameter of interest.
\begin{itemize}
\item Simplest is the fractional uncertainty $\sigma_x/\langle x\rangle$, where $\sigma_x$ and $\langle x\rangle$ are the standard deviation and mean of the distributions for parameter $x$ respectively. This is particularly useful for showing how uncertainty scales with S/N: in the limit of high S/N, the standard deviation can be approximated from the (inverse) Fisher matrix and scales inversely with the S/N \citep{Vallisneri_2008}.
\item The credible interval $\mathrm{CI}_p^{x}$ in the range that contains the central $p$ of the integrated posterior, with $(1-p)/2$ falling both above and below the limits \citep{Aasi_2013}. Specifying the credible interval for several values of $p$ gives information about the shape of the posterior.
\item As an alternative to credible intervals, we use credible upper or lower bounds. These are the one-sided equivalents of credible intervals, and are useful for distributions that are peaked towards one end of the parameter range or for parameters we are interested in putting a limit upon (the spin magnitude satisfies both of these criteria).
\item For sky-localization, we use credible regions (the two-dimensional generalization of the credible interval) which are the smallest sky area that encompasses a given total posterior probability. The credible region for a total posterior probability $p$ is defined as
\begin{equation}
\mathrm{CR}_p = \underset{A}{\arg\!\max} \int_A \mathrm{d}\boldsymbol{\Omega} P_{\Omega}(\boldsymbol{\Omega}),
\label{eq:CR}
\end{equation}
where $P_{\Omega}(\boldsymbol{\Omega})$ is the posterior PDF over sky position $\boldsymbol{\Omega}$, and $A$ is the sky area integrated over \citep{Sidery_2014}. We also consider the searched area $A_\ast$, the area of the smallest credible region that includes the true location.
\end{itemize}
  
To check that differences between our spinning and non-spinning analyses were a consequence of the inclusion of spin and not because of a difference between waveform approximants, we also ran SpinTaylorT4 analyses with spins fixed to $\chi_{1,~2}=0$. There were no significant differences in parameter estimation between the non-spinning TaylorF2 and zero-spin SpinTaylorT4 results for any of the quantities we examined.\footnote{Using as an example the chirp mass, the most precisely inferred parameter, we can compare the effects of switch from a non-spinning to a spinning analysis to those from switching waveform approximants by comparing the difference the posterior means $\langle \mathcal{M}_\mathrm{c}\rangle$. The difference between means from the SpinTaylorT4 analyses with and without spin, is an order of magnitude greater than the difference between the zero-spin SpinTaylorT4 and TaylorF2 analyses: defining the log-ratio $\xi = \log_{10}(|\langle \mathcal{M}_\mathrm{c}\rangle^\mathrm{S} - \mathcal{M}_\mathrm{c}\rangle^0|/|\langle \mathcal{M}_\mathrm{c}\rangle^\mathrm{NS} - \mathcal{M}_\mathrm{c}\rangle^0|)$, where the superscripts $\mathrm{S}$, $0$ and $\mathrm{NS}$ indicates results of the fully spinning SpinTaylorT4, the zero-spin SpinTaylorT4 and the non-spinning TaylorF2 analyses respectively, the mean (median) value of $\xi$ is $0.90$ ($1.04$), and $92.4\%$ of events have $\xi > 0$ (indicating that the shift in the mean from introducing spin is larger than the shift from switching approximants).} Therefore, we only use the TaylorF2 results to illustrate the effects of neglecting spin.
  
  
  
  
  
  
  