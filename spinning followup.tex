\section{Spinning Analysis}
\label{sec:spin}

\citet{Singer_2014} details the detection, low-latency localization, and medium-latency (i.e. non-spinning) follow-up of the simulated signals in 2015. In this work we perform the expensive analysis of full parameter estimation that accounts for non-zero compact object spin. Whereas \citet{Singer_2014} used the (non-spinning) TaylorF2 waveform model, we make use of the SpinTaylorT4 waveform model \citep{Buonanno_2003,Buonanno_2009}, parameterized by the fifteen parameters that uniquely define a circularized compact binary inspiral.

The simulated population of BNS systems contains slowly spinning NSs, with masses between $1.2~\mathrm{M}_\odot$ and $1.6~\mathrm{M}_\odot$ and spin magnitudes $\chi < 0.05$.  This choice was motivated by the characteristics of NSs found thus far in BNS systems. However, neutron stars \emph{outside} of BNS systems have been observed with spins as high as $\chi = 0.4$ \citep{Hessels_2006,Brown_2012}, and depending on the neutron-star equation of state (EOS) could theoretically have spins as high as $\chi \lesssim 0.7$ \citep{Lo_2011} without breaking up.  For these reasons, the prior assumptions used for Bayesian inference of source parameters are more broad than the spin range of the simulated source population.

To simulate a real analysis scenario where the class of compact binary and the NS EOS will not be known, we use uniform priors in component masses between $0.6~\mathrm{M}_\odot$ and $5.0~\mathrm{M}_\odot$ to avoid any prior constraints on mass posteriors, and our standard spin prior: uniform in spin magnitudes $\chi_{1,2} \sim U(0, 1)$ and isotropic in spin orientation. Prior distributions for location and orientation of the binary match that of the simulated population, i.e., uniform in volume (out to a maximum distance of $218.9~\mathrm{Mpc}$, safely outside the detection horizon) and isotropically oriented.  Choosing any particular upper bound for spin magnitude would require either assuming hard constraints on NS spin-up, which are based upon observations with hard to quantify selection effects, or making assumptions regarding the unknown EOS of NSs. For these reasons we choose not to rule out compact objects with high spin a priori by using an upper limit of $\chi < 1$, encompassing all allowed NS and black hole (BH) spins.  In section \ref{subsec:prior_constraints} we look at more constraining spin priors, and particularly how such choices can affect mass estimates.
  
  
  
  
  
  