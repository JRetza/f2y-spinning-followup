\section{Spinning Analysis}

\citet{2013arXiv1304.0670L} details the detection, low-latency localization, and `medium-latency' follow-up of the simulated signals in 2015.  Missing from the study is the full characterization of the source, accounting for spin.  In this work we simulate this final step in the detection and characterization of BNS signals using the ``SpinTaylorT4'' fifteen-dimensional model for the GWs emmitted by generically spinning, cicularized compact binary mergers \cite{Buonanno_2003,Buonanno_2009}. 

The population chosen to be simulated for these studies, with spin magnitudes $\chi < 0.05$, have small enough spins that biases in parameter estimates due to assuming the components to be non-spinning are well below the statistical uncertainty (CITATION NEEDED).  This choice was motivated by the charactistics of neutron stars found in BNS systems to date, however nuetron stars \emph{outside} of BNS systems have been observed with spins as high as $\chi = 0.4$ \cite{Hessels_2006,Brown_2012}, and depending on the NS equation-of-state (EOS) could have spins as high as $\chi \lesssim 0.7$ \cite{Lo_2011}.

When analyzing a GW trigger, the Bayesian analyses use `uninformative' priors for the source's properties.  For some parameters, such as the location and orientation of the binary, uniformative priors (i.e., uniform in volume and isotropically oriented) are well-motivated.  For other parameters (e.g., component masses, spins), current observations and theory do not motivate any particular choice in prior.  For this study we use a prior distribution uniform in component masses, as in \citet{2013arXiv1304.0670L}, and spins with magnitudes uniformly distributed between 0 and 1 and isotropically oriented.  We make the choice to not rule out compact objects with high spin because, since choosing any particular upper bound for spin magnatude would require choices regarding the unknown EOS of neutron stars.  In section (REF) we look at more constraining spin priors, and particularly how such choices can affect mass estimates.

\subsection{Spinning parameter estimates}


\subsection{Comparison against non-spinning analsis}

\begin{enumerate}
\item Show how mass estimates compare.
\end{enumerate}


\subsection{Restricting Spin Priors}

\begin{enumerate}
\item Motivate restricting spin magnitude for NSs
\item Show spin must be \textit{very} restricted to affect mass estimates.
\end{enumerate}