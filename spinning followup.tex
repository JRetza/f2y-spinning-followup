\section{Spinning Analysis}

\citet{Singer_2014} details the detection, low-latency localization, and `medium-latency' follow-up of the simulated signals in 2015.  Missing from the study is the complete characterization of sources, accounting for the spins of the NSs and their degeneracies with other parameters.  In this work we simulate this final step in the detection and characterization of BNS signals using the ``SpinTaylorT4'' fifteen-dimensional model for the GWs emmitted by generically spinning, cicularized compact binary mergers \cite{Buonanno_2003,Buonanno_2009}. 

The simulated population of BNS systems contains very slowly spinning neutron stars, with spin magnitudes $\chi < 0.05$).  This choice was motivated by the charactistics of neutron stars found in BNS systems to date. However, nuetron stars \emph{outside} of BNS systems have been observed with spins as high as $\chi = 0.4$ \cite{Hessels_2006,Brown_2012}, and depending on the NS equation-of-state (EOS) could have spins as high as $\chi \lesssim 0.7$ \cite{Lo_2011} without breaking up.  Thus resticting priors to the very narrow spin range used for the simulated population, could result in significant biases in parameter estimates if neutron stars with higher spins exist.

Though the main purpose of this study is to quantify parameter estimates while accounting for spin, we will include two non-spinning analyses for comparison.  The first is the analysis using the ``TaylorF2'' frequency-domain model, the results of which were presented in \citet{Singer_2014}.  This model is very fast, cheap to generate, and is always used for medium-latency follow-up before more expensive analyses are done.  The second non-spinning analysis uses the time-domain ``SpinTaylorT4'' model, with spins fixed to 0.  This is not typically used in a follow-up scenario, but is included to demonstrate that any differences between the spinning and ``TaylorF2'' analyses are due to the inclusion of spin, and not systematic differences between waveform models.

When analyzing a GW trigger, the Bayesian analyses use `uninformative' priors for the source's properties.  For some parameters, such as the location and orientation of the binary, uniformative priors (i.e., uniform in volume and isotropically oriented) are well-motivated.  For other parameters (e.g., component masses, spins), current observations and theory do not motivate any particular choice in prior.  For this study we use a prior distribution uniform in component masses, as in \citet{2013arXiv1304.0670L}, and spins with magnitudes uniformly distributed between 0 and 1 and isotropically oriented.  We make the choice to not rule out compact objects with high spin because, since choosing any particular upper bound for spin magnatude would require choices regarding the unknown EOS of neutron stars.  In Sec. \ref{subsec:prior_constraints} we look at more constraining spin priors, and particularly how such choices can affect mass estimates.