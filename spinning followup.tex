\section{Spinning Analysis}
\label{sec:spin}

\citet{Singer_2014} details the detection, low-latency localization, and medium-latency (i.e. non-spinning) follow-up of the simulated signals in 2015. This work is the final step of BNS characterization for the same data. We make use of the SpinTaylorT4 waveform model \citep{Buonanno_2003,Buonanno_2009}, parameterized by the fifteen parameters that uniquely define a circularized compact binary merger.

The simulated population of BNS systems contains slowly spinning neutron stars, with spin magnitudes $\chi < 0.05$.  This choice was motivated by the characteristics of neutron stars found thus far in BNS systems. However, neutron stars \emph{outside} of BNS systems have been observed with spins as high as $\chi = 0.4$ \citep{Hessels_2006,Brown_2012}, and depending on the neutron-star equation of state (EOS) could have spins as high as $\chi \lesssim 0.7$ \citep{Lo_2011} without breaking up.  For these reasons, the prior assumptions used for Bayesian inference of source parameters is more broad than the spin range of the simulated source population.  To simulate a real detection scenario where the types of compact objects in the binary and the neutron star EOS will not be known, we use our `standard' spin prior that is uniform in spin magnitudes $\chi_{1,2} \sim U(0, 1)$ and isotropic in spin orientation. Prior distributions for location and orientation of the binary match that of the simulated population (i.e. uniform in volume and isotropically oriented).  We chose not to rule out compact objects with high spin a priori because using any particular upper bound for spin magnitude would require either assuming hard constraints on neutron-star spin-up, which are based upon observations with potential selection effects, or making assumptions regarding the unknown EOS of neutron stars. Thus, we use an upper limit of $\chi < 1$ which safely encompasses all allowed neutron-star and black-hole spins.  In section \ref{subsec:prior_constraints}, we look at more constraining spin priors, and particularly how such choices can affect mass estimates.