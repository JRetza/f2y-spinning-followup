We can consider sky localization in greater detail by comparing areas on an event-by-event basis and not just the cumulative distribution across the population. Doing this, we confirm that sky localization is consistent between approaches for any given event. We use the medium-latency TaylorF2 analysis as a reference point, and compare the ratio of sky areas. To summarise the variation in sky areas computed in different analyses, we use the log ratio
\begin{equation}
\mathcal{R}_A^X = \log_{10}\left(\frac{A^X}{A^\mathrm{TF2}}\right)
\end{equation}
where $A^X$ is a credible region or the searched area as determined by method $X$ and $A^\mathrm{TF2}$ is the same quantity from the TaylorF2 analysis. The log ratio $\mathcal{R}_A^X$ is zero when analysis $X$ agrees with the TaylorF2 results. Considering all $250$ events, the mean and standard deviation of the log ratio is given in Table~\ref{tab:sky-ratio}. There is is no significant difference between analyses. The computationally expensive fully spinning analysis does not improve sky localization: there is no disadvantage in using the lower latency results for EM follow-up.