\begin{abstract}
Inspiraling binary neutron stars are expected to be one of the most significant sources of gravitational-wave signals for the new generation of advanced ground-based detectors. Advanced LIGO \textbf{began} operation in 2015 and we investigate how well we could hope to measure properties of these binaries should a detection be made in the first observing period. We study an astrophysically motivated population of sources (binary components with masses $1.2~\mathrm{M}_\odot$--$1.6~\mathrm{M}_\odot$ and spins of less than $0.05$) using the full LIGO analysis pipeline. While this simulated population covers the observed range of potential binary neutron-star sources, we do not exclude the possibility of sources with parameters outside these ranges; given the existing uncertainty in distributions of mass and spin, it is critical that analyses account for the full range of possible mass and spin configurations. We find that conservative prior assumptions on neutron-star mass and spin lead to average fractional uncertainties in component masses of $\sim 16\%$, with little constraint on spins (the median $90\%$ upper limit on the spin of the more massive component is $\sim 0.7$).  Stronger prior constraints on neutron-star spins can further constrain mass estimates, but only marginally.  However, we find that the sky position and luminosity distance for these sources are not influenced by the inclusion of spin; therefore, if LIGO detects a low-spin population of BNS sources, less computationally expensive results calculated neglecting spin will be sufficient for guiding electromagnetic follow-up.
\end{abstract}

\keywords{gravitational waves -- methods: data analysis -- stars: neutron -- surveys}
  
  
  
  
  
  
  
  