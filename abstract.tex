\begin{abstract}
Inspiraling binary neutron stars are expected to be one of the most significant sources of gravitational-wave signals for the new generation of advanced ground-based detectors. Advanced LIGO will begin operation in 2015 and we investigate how well we could hope to measure properties of these binaries should a detection be made in the first observing period. Given the existing uncertainty in distributions of mass and spin for this population, it is critical that analyses account for the full range of possible mass and spin configurations. Considering an astrophysically motivated population of sources (binary components with masses $1.2~\mathrm{M}_\odot$--$1.6~\mathrm{M}_\odot$ and low spins of less than $0.05$) and the full LIGO analysis pipeline, conservative prior assumptions on neutron star mass and spin lead to average fractional uncertainties in component masses of $\sim 16\%$, with little constraint on spins.  Stronger prior constraints on neutron-star spins can further constrain mass estimates, but only marginally.  However, we find that the sky position and luminosity distance are not influenced by the inclusion of spin; therefore, less computationally expensive results calculated neglecting spin can be used with impunity for electromagnetic follow-up.
\end{abstract}

\keywords{gravitational waves -- methods: data analysis -- stars: neutron -- surveys}
  
  
  
  