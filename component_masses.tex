Figure \ref{fig:comp_masses} compares cartoon $90\%$ credible regions in component-mass space of $5$ chosen simulated signals \citep[cf.][figure 1]{Chatziioannou_2014}.  As a consequence of the difficulty of estimating the narrow and nonlinearly correlated credible regions in $m_1$--$m_2$ space, we illustrate the credible regions in $m_1$--$m_2$ space as the projection of a rectangular region in $\mathcal{M}_\mathrm{c}$--$q$ space.  \textbf{To define the rectangular region we use $90\%$ credible intervals of the one-dimensional posterior PDFs of $\mathcal{M}_\mathrm{c}$ and $q$; for $\mathcal{M}_\mathrm{c}$ we use the central $90\%$ credible interval ($5$th to $95$th percentile), and for $q$ the upper $90\%$ credible interval ($10$th to $100$th percentile).  These differing credible intervals were chosen to better summarize the one-dimensional posterior PDFs, which are typically normal for $\mathcal{M}_\mathrm{c}$ and skewed toward high values for $q$ (see figure \ref{fig:mass_pdfs}).} It is evident that only fairly strong prior assumptions on NS spin ($\chi \lesssim 0.4$) will significantly impact mass constraints.