Figure \ref{fig:comp_masses} compares cartoon $90\%$ credible regions in component-mass space of $5$ chosen simulated signals \citep[cf.][figure 1]{Chatziioannou_2014}.\footnote{As a consequence of the difficulty of estimating the narrow and nonlinearly correlated credible regions in $m_1$--$m_2$ space, we illustrate the credible regions in $m_1$--$m_2$ space as the projection of a rectangular region in $\mathcal{M}_\mathrm{c}$--$q$ space, bounded in chirp-mass by the central $90%$ credible interval ($5$th to $95$th percentile) and in mass-ratio by the upper $90%$ credible interval ($10$th to $100$th percentile).} If the analyses used a prior distribution for spin matching the simulated population ($\chi < 0.05$), we would expect the masses of the simulated sources to fall within the estimated $90\%$ credible regions for close to $90\%$ of simulations. For the broad spin prior used ($\chi < 1$) however, this will not be the case.  For this reason, several of the simulated sources lie outside of the $90\%$ credible regions in figure \ref{fig:comp_masses}.  It is evident that overly conservative prior assumptions about NS spin ($\chi < 1$) will \textit{strongly} affect uncertainties in mass estimates.
  
  
  
  
  
  
  