\subsection{Prior Constraints on Spin}
\label{subsec:prior_constraints}

Since spin is largely degenerate with mass ratio, and spin is expected to be small for BNS sources, it is interesting to ask how the mass constraints are affected by making stronger prior assumptions about the spin of NSs.  First, we make the extreme assumption that NSs have negligible spin, as was done in \citet{Singer_2014} and \citet{Berry_2014}.  Figure \ref{fig:mass_std} compares the distribution of (fractional) uncertainties in chirp-mass and mass-ratio estimates for the spinning and non-spinning analyses.\footnote{We only show results from the TaylorF2 non-spinning analyses, but we also ran SpinTaylorT4 analyses with spins fixed to $\chi_{1,~2}=0$ and validated that systematic differences in waveform models do not significantly affect estimates. Using as an example the chirp mass, the most precisely inferred parameter, we can compare the effects of switch from a non-spinning to a spinning analysis to those from switching waveform approximants by comparing the difference the posterior means $\langle \mathcal{M}_\mathrm{c}\rangle$. The difference between means from the SpinTaylorT4 analyses with and without spin, is an order of magnitude greater than the difference between the non-spinning SpinTaylorT4 and TaylorF2 analyses. There were no significant differences in parameter estimation between the non-spinning TaylorF2 and non-spinning SpinTaylorT4 results for any of the quantities we examined.} The mean (median) fractional uncertainties in chirp mass and mass ratio are $0.0185\%$ ($0.0165\%$) and $8.90\%$ ($8.79\%$), respectively, both being a factor of $\sim3$--$4$ smaller than the uncertainties from a spinning analysis.
  
  
  
  
  
  
  
  
  