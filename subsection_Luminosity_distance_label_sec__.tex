\subsection{Luminosity distance}\label{sec:distance}

The distance is degenerate with the inclination \citep{Cutler_1994,Aasi_2013}, and the inclination can be better constrained for precessing systems \citep{van_der_Sluys_2008,Vitale_2014}. Since we are considering a population with low spins, precession is minimal, and there should be little effect from including spin in the analysis.

The absolute size of the distance credible interval $\mathrm{CI}_p^{D}$ approximately scales with the distance, hence we divide the credible interval by the true (injected) distance $D_\star$; this gives an approximate analogue of twice the fractional uncertainty \citep{Berry_2014}. The cumulative distribution of the scaled credible intervals is plotted in Figure \ref{fig:distance}. The mean (median) values of $\mathrm{CI}_{0.5}^{D}/D_\star$ for the spinning and non-spinning analyses are $0.436$ ($0.376$) and $0.426$ ($0.363$) respectively; the values of $\mathrm{CI}_{0.9}^{D}/D_\star$ are $0.981$ ($0.845$) and $0.951$ ($0.819$), and the fractional uncertainties $\sigma_D/\langle D\rangle$ are $0.302$ ($0.262$) and $0.245$ ($0.239$). There is negligible difference between the spinning and non-spinning analyses as expected.


  
  
  
  
  