\subsection{Luminosity distance}\label{sec:distance}

The distance is degenerate with the inclination \citep{Cutler_1994,Aasi_2013}, and the inclination can be better constrained for precessing systems \citep{van_der_Sluys_2008,Vitale_2014}. Since we are considering a population with low spins, precession is minimal, and there should be little effect from including spin in the analysis.

We quantify distance measurement accuracy using symmetric credible intervals: the distance credible interval $\mathrm{CI}_p^{D}$ in the range that contains the central $p$ of the integrated posterior, with $(1-p)/2$ falling both above and below the limits \citep{Aasi_2013}. The absolute size of the credible interval approximately scales with the distance, hence we divide the credible interval by the true (injected) distance $D_\star$; this gives an approximate analogue of twice the fractional uncertainty \citep{Berry_2014}. The cumulative distribution of the scaled credible intervals is plotted in Figure \ref{fig:distance}. There is negligible difference between the spinning and non-spinning analyses as expected.


  
  