In this study we investigated the effect of including spin when considering PE for BNS binaries with aLIGO. We expect neutron stars to be only slowly spinning, and hence that their spins only have a small effect of the GW signature of a BNS merger. However, introducing the component spins into PE has a significant effect. First, this requires the use of more computationally expensive waveforms (that include more physics) and so the latency of PE is increased. Second, the correlation of spin with other parameters affects the recovered PDFs.

The spins themselves are poorly constrained. There is a large uncertainty in most cases as indicated by some numbers.

The mass--spin degeneracy means posterior PDFs that are broader for the chirp mass and the mass ratio when spin is included. These posterior PDFs better represent what we believe about the parameters describing the source systems and so a broader posterior must not be viewed as degraded PE performance, rather more realistic.\footnote{We have now reliably measured that the glass is $50\%$ full.} Including spins broadens the chirp-mass distribution by a factor of two \citep{Poisson_1995,Berry_2014}, but the chirp-mass still remains precisely measured, the median something is $42$.

The sky-location, which is central to performing EM follow-up, is not correlated with the spin; it is unaffected by the inclusion (or exclusion) of spin in PE. The luminosity distance is similarly unaffected for this population of slowly spinning neutron stars. This means that the lower-latency results that could be supplied in time for EM observatories to search for a counter-part are already as good as we can hope to achieve with GW measurements.