In this study we investigated the effects of accounting for spin when estimating the parameters of BNS sources with aLIGO. We expect NSs to be only slowly spinning, and hence that their spins only have a small effect of the GW signature of a BNS merger. However, accounting for spins \emph{does} have a significant effect on parameter constraints. First, this requires the use of more computationally expensive waveforms (that include more physics), increasing the latency of analyses. Second, strong degeneracies are present in the model; not only are the spins themselves poorly constrained, these degeneracies result in weaker constraints on other parameters, particularly masses.

It is important to note that these weaker constraints are the result of accounting for our prior assumptions on NS spins.  We tested various choices for conservative prior assumptions about NS spins and found them to have little effect on mass estimates.  Only very strong prior assumptions, such as say $\chi_{1,~2}\lesssim .05$ (consistent with the simulated population, and NSs found in binaries to date) are likely to significantly affect mass constraints.  Such prior assumptions will be hard to justify, however, based on the small number of observed systems and unknown selection effects. 

The sky-location, which is central to performing EM follow-up, is not correlated with the spin; it is unaffected by the inclusion (or exclusion) of spin in parameter estimation. The luminosity distance is similarly unaffected for this population of slowly spinning NSs. This means that the lower-latency results that could be supplied in time for EM observatories to search for a counter-part are already as good as we can hope to achieve with GW measurements.
  
  