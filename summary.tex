\section{Conclusions}\label{sec:conclusions}

In this study we investigated the effects of accounting for spin when estimating the parameters of BNS sources with aLIGO. We expect NSs to be only slowly spinning, and hence that their spins only have a small effect of the GW signature of a BNS merger. However, allowing for spins \textit{does} have a significant effect on parameter constraints. Strong degeneracies are present in the model; not only are the spins themselves poorly constrained, but these degeneracies result in weaker constraints on other parameters, particularly masses.  Excluding spin from parameter estimation results in artificially narrow and potentially inaccurate posterior distributions.

Weaker constraints are the result of accounting for broad prior assumptions on NS spins.  We tested various choices for conservative prior assumptions about NS spins and found them to have little effect on mass estimates.  Only strong prior assumptions, such as say $\chi_{1,~2}\lesssim 0.05$ (consistent with the simulated population, and NSs observed in short-period BNS binaries to date) are likely to significantly affect mass constraints.  However, such strict prior assumptions are hard to justify given the small number of observed systems and possible selection effects.

We performed parameter estimation on a astrophysically motivated population of BNS signals, assuming an aLIGO sensitivity comparable to that expected \textbf{throughout} its first observing run. Using a prior on spin magnitudes that is uniform from $0$ to $1$, spanning the range permitted for BHs and extending beyond the expected (but uncertain) upper limit for NSs, the median $90\%$ upper limit on the spin of the more massive component is $0.70$ and the limit for the less massive component is $0.86$. The median fractional uncertainty for the mass ratio $\sigma_q/\langle q \rangle$ is $\sim30\%$, the median fractional uncertainty for the total mass $\sigma_{{M}}/\langle {M} \rangle$ is $\sim6\%$ and the median fractional uncertainty for the chirp mass $\sigma_{\mathcal{M}_\mathrm{c}}/\langle {\mathcal{M}_\mathrm{c}} \rangle$ is $\sim0.06\%$. Despite the mass--spin degeneracy and only weak constraints on the spin magnitudes, we find that we can place precise constraints on the chirp mass for these BNS signals.

The sky-location accuracy, which is central to performing EM follow-up, is not affected by including spin in the analysis of low-spin systems; this may not be the case when spin is higher, i.e.\ in binaries containing a BH. For our population of BNSs, sky localization is unchanged by the inclusion (or exclusion) of spin in parameter estimation. The median $\mathrm{CR}_{0.9}$ ($\mathrm{CR}_{0.5}$) is $\sim 500~\mathrm{deg^2}$ ($\sim 130~\mathrm{deg^2}$). The luminosity distance is similarly unaffected for this population of slowly spinning NSs; the median fractional uncertainty $\sigma_D/\langle D \rangle$ is $\sim 25\%$.  However, an analysis that includes spins requires the use of more computationally expensive waveforms (that include more physics), increasing latency by an order of magnitude.  Therefore, if the population matches our current expectation of being slowly spinning, the low-latency results that could be supplied in time for EM observatories to search for a counter-part are as good as the high-latency results in this respect, and there is no benefit in waiting.

\textbf{Following the submission of this article, aLIGO made its first detection \citep{Abbott:2016blz}. This was of a binary BH system \citep{TheLIGOScientific:2016wfe} rather than a BNS, but much of our understanding of the abilities of the parameter-estimation analysis, such as the effects of mass--spin degeneracy, translates between sources. The era of GW astronomy has begun, and parameter estimation will play a central role in the science to come.}