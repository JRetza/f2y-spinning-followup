\section{Source Simulation and Selection}

We have restricted our simulation to the first year of the advanced detector era, using the 2015 dataset from \citet{Singer_2014}.  For this, Gaussian noise was generated using the `early' 2015 LIGO noise curve found in \citep{Barsotti:2012}.  Approximately 50,000 BNS sources were simulated with component masses uniformly distributed between $1.2~\mathrm{M}_\odot$ and $1.6~\mathrm{M}_\odot$.  Spins were isotropically oriented, with magnitudes $\chi = c |\mathbf{S}|/G m^2$, where $|\mathbf{S}|$ is the neutron star's spin angular momentum and $m$ its mass, were drawn uniformly between $0$ and $0.05$.  The range of simulated spin magnitudes was chosen to be consistent with the observed population of BNS systems, currently bounded by PSR J0737$-$3039A \citep{Burgay_2003,Brown_2012}.  Finally, sources were distributed uniformly in volume (i.e., uniform in distance cubed).

From this simulated dataset, detected sources were selected using the \textsc{gstlal\_inspiral} matched-filter detection pipeline \citep{Cannon_2012} with a single-detector signal-to-noise ratio (S/N) theshold $\rho>4$, false alarm rate (FAR) threshold of $\mathrm{FAR}<10^{-2}~\mathrm{yr}^{-1}$.  The FAR for real detector noise is largely governed by glitches -- non-stationary noise transients -- in the data that can mimic GWs from compact binary mergers.  Because our noise is purely stationary and Gaussian (and therefore glitch-free), FAR estimates are overly optimistic, and an additional theshold on the network S/N $\rho_\mathrm{net} > 12$ is consistant with the above FAR theshold when applied to data similar to previous science runs \cite{2013arXiv1304.0670L}.  For more details regarding the simulated data and \textsc{gstlal\_inspiral} analyses, please see \citet{Singer_2014}.