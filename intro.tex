\begin{abstract}
Inspiraling binary neutron stars are expected to be the main source of gravitational-wave signals for the new generation of advanced ground-based detectors. Advanced LIGO will begin operation in 2015 and we investigate how well we could hope to measure properties of these binaries should a detection be made in the first observing period. To measure the masses and spins of the neutron stars accurately, it is essential to include the spins in the parameter estimation analysis. This makes parameter estimation more computationally expensive. Considering an astrophysically motivated population of sources, we find that the masses and spins are \ldots Even though our population is only slowly rotating, the mass--spin degeneracy impacts our results. However, extrinsic parameters, specifically the sky position and luminosity distance, and not influenced by the inclusion of spin; therefore, less computationally expensive results calculated neglecting spin can be used with impunity for electromagnetic follow-up.
\end{abstract}

\keywords{gravitational waves -- methods: data analysis -- stars: neutron -- surveys}

\section{Introduction}

As we prepare to enter the advanced-detector era of ground-based gravitational-wave (GW) astronomy, it is critical that we understand the abilities and limitations of the analyses we intend to conduct. Of the many predicted sources of GWs, binary neutron star (BNS) coalescences are paramount; their progenitors have been directly observed, and the advanced detectors will be sensitive to their GW emission up to $\sim 400~\mathrm{Mpc}$ away \citep{2013arXiv1304.0670L}.

When analyzing a GW signal from a circularized compact binary merger, strong degeneracies exist between parameters describing the binary (e.g., distance and inclination). To properly estimate any particular parameter(s) interest, the marginal distribution is calculated by integrating the joint posterior probability density function (PDF) over the remaining parameters. In this work, we use nested sampling \citep{Veitch_2010} and Markov-chain Monte Carlo \citep{Christensen_2003,R_ver_2006,van_der_Sluys_2008} techniques to sample the posterior PDF.

Previous studies of BNS signals have largely restricted parameter estimates of BNS signals to nine parameters by ignoring the spin of the compact objects. This simplification has largely been due to computational constraints, but the slow spin of neutron stars observed to date \citep[e.g.,][]{Mandel_2010} has also been used for justification. However, proper characterization of compact binary sources \emph{must} account for the possibility of non-zero spin, otherwise parameter estimates will be biased.  This bias can potentially lead to incorrect conclusions about source properties, and even misidentification of source classes. \citep{Buonanno_2009,Berry_2014}.

Numerous studies have looked at the BNS parameter-estimation abilities of ground-based GW detectors such as the Advanced Laser Interferometer Gravitational-Wave Observatory \citep[aLIGO;][]{Aasi_2015} and Advanced Virgo \citep[AdV;][]{Acernese_2014} detectors. \citet{Nissanke_2010,Nissanke_2011} assessed localization abilities on a simulated non-spinning BNS population. \citet{Veitch_2012} looked at several potential advanced-detector networks and quantified the parameter estimation abilities of each network for a fiducial non-spinning BNS signal. \citet{Aasi_2013} demonstrated the ability to characterize non-spinning BNS signals using spinning waveforms using Bayesian stochastic samplers in the \textsc{LALInference} library \citep{Veitch_2014}.  \citet{Hannam_2013} used approximate methods to quantify the degeneracy between spin and mass estimates, assuming the compact objects' spin is aligned with the orbital angular momentum. \citet{Rodriguez_2014} simulated a collection of loud, non-spinning BNS signals in several mass bins and quantified parameter-estimation capabilities in the advanced-detector era using non-spinning models.  Finally, \citet{Singer_2014} and the follow-on \citet{Berry_2014} represent an (almost) complete end-to-end simulation of BNS detection and characterization during the first $1$--$2$ years of the advanced-detector era. These studies simulated the GWs of a BNS population that was astrophysically motivated, detected and characterized sources using the detection and follow-up tools (reviewed by the LIGO--Virgo Collaboration) that are to be used in the coming years.   The final stage of the analysis missing from these studies is the computationally expensive characterization of sources while accounting for the compact objects' spins and their degeneracies with other parameters.  This work is the final step of BNS characterization for the \citet{Singer_2014} simulations using waveforms that account for the effects of neutron star spin.