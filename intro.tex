\section{Introduction}

As we prepare to enter the advanced-detector era of ground-based gravitational-wave (GW) astrononmy, it is critical that we understand the abilities and limitations of the analyses we are prepared to conduct.  Of the many predicted sources of GWs, binary neutron star (BNS) coelescences are paramount; their progenitors have been directly observed, and the advanced detectors will be sensitive to their GW emmission up to 400 Mpc away \cite{2013arXiv1304.0670L}.

Numerous studies have looked at the BNS parameter estimation abilities of ground-based GW detectors. \citet{Nissanke_2010,Nissanke_2011} assessed localization abilities on a simulated non-spinning BNS population.  \citet{Aasi_2013} demonstated the ability to characterize non-spinning BNS signals using spinning waveforms, and Bayesian parameter estimation (PE) tools in the \textsc{LALInference} library (CITE).   \citet{Hannam_2013} use appoximate methods to estimate confidence regions of parameter estimate probility distributions using and assess the ability \citet{Rodriguez_2014,Singer_2014}.

\begin{enumerate}
\item Highlight the importance of BNS signals for aLIGO
\item Mention the importance of spinning analyses
\end{enumerate}