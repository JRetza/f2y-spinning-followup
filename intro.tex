\section{Introduction}

As we prepare to enter the advanced-detector era of ground-based gravitational-wave (GW) astronomy, it is critical that we understand the abilities and limitations of the analyses we are prepared to conduct.  Of the many predicted sources of GWs, binary neutron star (BNS) coelescences are paramount; their progenitors have been directly observed, and the advanced detectors will be sensitive to their GW emmission up to 400 Mpc away \cite{2013arXiv1304.0670L}.

To properly estimate the parameters of a circularized compact binary merger, analyses must estimate all parameters simultaniously in order to account for the strong degeneracies between parameters.  This is often done using Bayesian samplers, particularly nested sampling \cite{Veitch_2010} and Markov-chain Monte Carlo \cite{Christensen_2003,R_ver_2006,van_der_Sluys_2008} techniques, that sample the fifteen-dimensional posterior probability density function.  Marginalization (i.e., integration) over some parameters then yield estimates for parameters of interest.

Historically it has been common to restrict parameter estimates of BNS signals to nine parameters by ignoring the spin of the compact objects.  This is partly motivated by the slow spin (relativistically speaking) of neutron stars observed to date, and partly due to the computational expense of completing such an analsis, espectially over a population of sources.  However, proper characterization of BNS sources \emph{must} account for the spin of the compact objects, otherwise parameter estimates will be biased, and potentially lead us to the wrong conclusions.

Numerous studies have looked at the BNS parameter estimation abilities of ground-based GW detectors. \citet{Nissanke_2010,Nissanke_2011} assessed localization abilities on a simulated non-spinning BNS population.  \citet{Veitch_2012} looked at several potential advanced-detector networks and quantified the PE abilities of each network for a fiducial non-spinning BNS signal.  \citet{Aasi_2013} demonstated the ability to characterize non-spinning BNS signals using spinning waveforms, and Bayesian parameter estimation (PE) tools in the \textsc{LALInference} library \cite{Veitch_2014}.   \citet{Hannam_2013} use appoximate methods to quantify the degeneracy between spin and mass estimates using spin-aligned models.  \citet{Rodriguez_2014} simulated a collection of loud, non-spinning BNS signals in several mass bins and quantified parameter estimation capabilities in the advanced-detector era using non-spinning models.  Finally, \citet{Singer_2014} completed an (almost) complete end-to-end simulation of BNS detection and characterization during the first two years of the advanced detecor era, simulating an astrophysically motivated population of BNS signals that were then detected and characterized using the tools that will be deployed to do so in the coming years.  This study is the final step of BNS characterization using the \citet{Singer_2014} simulations: full characterization of the BNS sources, accounting for the spin of the neutron stars.