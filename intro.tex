\section{Introduction}

As we prepare to enter the advanced-detector era of ground-based gravitational-wave (GW) astronomy, it is critical that we understand the abilities and limitations of the analyses we intend to conduct.  Of the many predicted sources of GWs, binary neutron star (BNS) coalescences are paramount; their progenitors have been directly observed, and the advanced detectors will be sensitive to their GW emission up to 400 Mpc away \citep{2013arXiv1304.0670L}.

When analyzing a GW signal from circularized compact binary merger, strong degeneracies exist between parameters describing the binary (e.g., distance and the binary's inclination).  Thus, to properly estimate parameters of interest, such as the masses of the binary's components, the remaining parameters of the full fifteen-dimensional posterior probability density function (PDF) must be marginalized over.  In this work, we use nested sampling \citep{Veitch_2010} and Markov-chain Monte Carlo \citep{Christensen_2003,R_ver_2006,van_der_Sluys_2008} techniques to sample the posterior PDF.

Historically, it has been common to restrict parameter estimates of BNS signals to nine parameters by ignoring the spin of the compact objects.  This is partly motivated by the slow spin (relativistically speaking) of neutron stars observed to date \citep[e.g.,][]{Mandel_2010}, and partly due to the computational expense of completing such an analysis, especially over a population of sources.  However, proper characterization of BNS sources \emph{must} account for the spin of the compact objects, otherwise parameter estimates will be biased, potentially leading to incorrect conclusions about source properties, and even source class. \citep{Buonanno_2009,Berry_2014}.

Numerous studies have looked at the BNS parameter-estimation abilities of ground-based GW detectors such as the Advanced Laser Interferometer Gravitational-Wave Observatory \citep[aLIGO;][]{Aasi_2015} and Advanced Virgo \citep[AdV;][]{Acernese_2014} detectors. \citet{Nissanke_2010,Nissanke_2011} assessed localization abilities on a simulated non-spinning BNS population.  \citet{Veitch_2012} looked at several potential advanced-detector networks and quantified the PE abilities of each network for a fiducial non-spinning BNS signal.  \citet{Aasi_2013} demonstrated the ability to characterize non-spinning BNS signals using spinning waveforms, and Bayesian parameter estimation (PE) tools in the \textsc{LALInference} library \cite{Veitch_2014}.   \citet{Hannam_2013} use approximate methods to quantify the degeneracy between spin and mass estimates using spin-aligned models.  \citet{Rodriguez_2014} simulated a collection of loud, non-spinning BNS signals in several mass bins and quantified parameter-estimation capabilities in the advanced-detector era using non-spinning models.  Finally, \citet{Singer_2014} and the follow-on \citet{Berry_2014} represent an (almost) complete end-to-end simulation of BNS detection and characterization during the first $1$--$2$ years of the advanced-detector era, simulating an astrophysically motivated population of BNS signals that were then detected and characterized using the tools that will be deployed to do so in the coming years.   Missing from the \citet{Singer_2014} study is the complete characterization of sources, accounting for the spins of the neutron stars (NSs) and their degeneracies with other parameters.  This work is the final step of BNS characterization for the \citet{Singer_2014} simulations using waveforms that account for the effects of neutron star spin.