\section{Introduction}

As we prepare to enter the advanced-detector era of ground-based gravitational-wave (GW) astronomy, it is critical that we understand the abilities and limitations of the analyses we are prepared to conduct. Of the many predicted sources of GWs, binary neutron star (BNS) coalescences are paramount; their progenitors have been directly observed\cite{Lorimer_2008}, and the advanced detectors will be sensitive to their GW emission up to $\sim 400~\mathrm{Mpc}$ away \citep{2013arXiv1304.0670L}.

When analyzing a GW signal from a circularized compact binary merger, strong degeneracies exist between parameters describing the binary (e.g., distance, inclination). To properly estimate any particular parameter(s) of interest, the marginal distribution is estimated by integrating the joint posterior probability density function (PDF) over all other parameters. In this work, we use nested sampling \citep{Veitch_2010} and Markov-chain Monte Carlo \citep{Christensen_2003,R_ver_2006,van_der_Sluys_2008} techniques to sample the posterior PDF.

Previous studies of BNS signals have largely assessed parameter constraints assuming negligible neutron star (NS) spin, restricting models to nine parameters. This simplification has largely been due to computational constraints, but the slow spin of NSs observed to date \citep[e.g.,][]{Mandel_2010} has also been used for justification. However, proper characterization of compact binary sources \emph{must} account for the possibility of non-negligible spin, otherwise parameter estimates will be biased.  This bias can potentially lead to incorrect conclusions about source properties, and even misidentification of source classes \citep{Buonanno_2009,Berry_2014}.

Numerous studies have looked at the BNS parameter estimation abilities of ground-based GW detectors such as the Advanced Laser Interferometer Gravitational-Wave Observatory \citep[aLIGO;][]{Aasi_2015} and Advanced Virgo \citep[AdV;][]{Acernese_2014} detectors. \citet{Nissanke_2010,Nissanke_2011} assessed localization abilities on a simulated non-spinning BNS population. \citet{Veitch_2012} looked at several potential advanced-detector networks and quantified the parameter estimation abilities of each network for a signal from a fiducial BNS with non-spinning NSs. \citet{Aasi_2013} demonstrated the ability to characterize signals from non-spinning BNS sources with waveform models for spinning sources using Bayesian stochastic samplers in the \textsc{LALInference} library \citep{Veitch_2014}.  \citet{Hannam_2013} used approximate methods to quantify the degeneracy between spin and mass estimates, assuming the compact objects' spins are aligned with the orbital angular momentum of the binary. \citet{Rodriguez_2014} simulated a collection of loud signals from non-spinning BNS sources in several mass bins and quantified parameter estimation capabilities in the advanced-detector era using non-spinning models.  Finally, \citet{Singer_2014} and the follow-on \citet{Berry_2014} represent an (almost) complete end-to-end simulation of BNS detection and characterization during the first $1$--$2$ years of the advanced-detector era. These studies simulated the GWs from astrophysically motivated BNS population, then detected and characterized sources using the detection and follow-up tools (reviewed by the LIGO--Virgo Collaboration) that are to be used in the coming years.   The final stage of the analysis missing from these studies is the computationally expensive characterization of sources while accounting for the compact objects' spins and their degeneracies with other parameters.  This work is the final step of BNS characterization for the \citet{Singer_2014} simulations using waveforms that account for the effects of NS spin.

We begin with a brief introduction to the source catalog used for this study and \citet{Singer_2014} in section \ref{sec:sources}. Then, in section \ref{sec:spin} we describe the results of parameter estimation from an full analysis that includes spin. In section \ref{sec:mass} we look at mass estimates in more detail, and spin-magnitude estimates in Section \ref{sec:spin-magnitudes}. In section \ref{sec:extrinsic} we consider estimation of extrinsic parameters: sky position (section \ref{sec:sky}) and distance (section \ref{sec:distance}) which we do not expect to be significantly affected by the inclusion of spin. We summarize our findings in section \ref{sec:conclusions}. A comparison of computational costs for spinning and non-spinning parameter estimation is given in appendix \ref{ap:CPU}.
    
  
  
  