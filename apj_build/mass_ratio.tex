We can investigate the impact of stronger prior assumptions regarding the maximum spin of NSs on mass estimates by discarding posterior samples above a given spin.  Figure \ref{fig:restricted_priors} shows the cumulative distribution of lower $90\%$ bounds on the estimates of $m_2$ among the $250$ simulated sources for spin priors of $\chi_{1,~2} \leq \{1, 0.7, 0.4, 0\}$.  $\chi_{1,~2}<1$ and $\chi_{1,~2}=0$ correspond to the spinning and non-spinning analyses described above.  $\chi<0.7$ is consistent with the NSs remaining intact for most proposed non-exotic EOSs.  $\chi<0.4$ is consistent with the spin of observed, isolated NSs to date.

From these PDFs, it is clear that fairly strong prior assumptions on NS spin are required to significantly impact mass constraints. Assuming NSs to be spinning with $\chi_{1,~2}\leq 0.4$ a priori only constrains masses by an extra few percent compared to allowing them to have $\chi_{1,~2} \leq 1$.