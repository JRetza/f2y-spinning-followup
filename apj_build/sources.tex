\section{Source Simulation and Selection}\label{sec:sources}

We have restricted our study to the first year of the advanced-detector era, using the same $250$ simulations that \citet{Singer_2014} analysed with non-spinning parameter estimation. For these, Gaussian noise was generated using the `early' 2015 aLIGO noise curve found in \citet{Barsotti:2012}. Approximately $50,000$ BNS sources were simulated, using the SpinTaylorT4 waveform model \citep{Buonanno_2003,Buonanno_2009}, a post-Newtonian inspiral model that includes the effects of precession, to generate the GW signals. Component masses were uniformly distributed between $1.2~\mathrm{M}_\odot$ and $1.6~\mathrm{M}_\odot$, which reflects the range of observed BNS masses \citep{_zel_2012}. Component spins were isotropically oriented, with magnitudes $\chi_{1,\,2} = c |\boldsymbol{S}_{1,\,2}|/G m_{1,\,2}^2$ drawn uniformly between $0$ and $0.05$; here $|\boldsymbol{S}_{1,\,2}|$ are the NSs' spin angular momenta and $m_{1,\,2}$ their mass, the indices $1$ and $2$ correspond to the more and less massive components of the binary, respectively.  The range of simulated spin magnitudes was chosen to be consistent with the observed population of short-period BNS systems, currently bounded by PSR J0737$-$3039A \citep{Burgay_2003,Brown_2012} from above.  Finally, sources were distributed uniformly in volume (i.e. uniform in distance cubed) to a maximum distance at which the loudest signal would produce a network signal-to-noise ratio (S/N) of $\rho_\mathrm{net} = 5$ \citep{Singer_2014}, where $\rho_\mathrm{net} = \sum_i \rho_i^2$ is the individual detector S/Ns $\rho_i$ combined in quadrature.

Of this simulated population, detectable sources were selected using the \textsc{gstlal\_inspiral} matched-filter detection pipeline \citep{Cannon_2012} with a single-detector S/N threshold $\rho>4$ and false alarm rate (FAR) threshold of $\mathrm{FAR}<10^{-2}~\mathrm{yr}^{-1}$.  The FAR for real detector noise is largely governed by non-stationary noise transients in the data that can mimic GWs from compact binary mergers, which \citet{Berry_2014} demonstrate make negligible difference to parameter estimation for the (low-FAR, BNS) signals considered here.  Because our simulated noise is purely stationary and Gaussian with no such artifacts, FAR estimates are overly optimistic. To compensate, an additional threshold on the network S/N of $\rho_\mathrm{net} > 12$ was applied. This S/N threshold is consistent with the above FAR threshold when applied to data similar to previous science runs \citep{2013arXiv1304.0670L,Berry_2014}. A random subsample of $250$ detections were selected for parameter estimation with \textsc{LALInference}.\footnote{The mean (median) $\rho_\mathrm{net}$ of the set of $250$ events is $16.7$ ($14.6$).} This mass and spin distributions of this subset is statistically consistent with those the sources were drawn from \citep{Berry_2014}. See \citet{Singer_2014} for more details regarding the simulated data and \textsc{gstlal\_inspiral} analyses.
  