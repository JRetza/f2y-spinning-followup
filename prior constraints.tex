It is physically impossible for NSs to have extremal spins of $\chi = 1$ however, and it is clear that prior assumptions about spin can affect mass estimates.  By discarding posterior samples above a given spin, we can effectively make stronger prior assumptions and see how mass estimates are affected.  Figure \ref{fig:restricted_priors} shows the average posterior mass ratio PDFs for maximum spins of $\chi < \{1, 0.7, 0.4, 0.2, 0.1, 0\}$.  $\chi<1$ and $\chi<0$ correspond to the spinning and non-spinning analyses looked at thus far.  $\chi<0.7$ is constistent with the NSs remaining entact based on proposed equations-of-state.  $\chi<0.4$ is constistent with the spin of observed, isolated NSs to date.  $\chi<0.2$ and $\chi<0.1$ are arbitrarely chosen points to show the evolution of the PDF toward the non-spinning case.

From these PDFs, it is clear that \emph{very} strong prior constraints have to be placed in order to have measurable effects on mass estimates.  Most would argue whether prior constraints on spin should be consistent with the observed NS population, or theoretical equations of state.  The main argument for not making the most conservative assumpions is that it will limit mass constraints.  However, these results show that even allowing spins well above break-up will not have much impact on mass constraints if the NS population turns out to be slowly-spinning.