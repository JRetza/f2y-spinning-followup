Prior assumptions about spin can affect mass estimates.  By discarding posterior samples above a given spin, we can effectively make stronger prior assumptions to see how mass estimates are affected.  Figure \ref{fig:restricted_priors} shows the average posterior mass-ratio PDFs for maximum spins of $\chi \leq \{1, 0.7, 0.4, 0.2, 0.1, 0\}$.  $\chi<1$ and $\chi=0$ correspond to the spinning and nonspinning analyses looked at thus far.  $\chi<0.7$ is consistent with the NSs remaining intact based on proposed EOSs.  $\chi<0.4$ is consistent with the spin of observed, isolated NSs to date.  $\chi<0.2$ and $\chi<0.1$ are arbitrarily chosen to show the evolution of the PDF toward the nonspinning case.

From these PDFs, it is clear that \emph{very} strong prior constraints have to be placed in order to have measurable effects on mass estimates. One could argue whether prior constraints on spin should be consistent with the observed NS population, or theoretical EOS predictions.  The main argument against using the most conservative assumptions is that this could limit mass constraints.  However, figure \ref{fig:restricted_priors} shows that even extending the prior well above the break-up limit does have a significant impact on mass constraints if the NS population turns out to be slowly-spinning.