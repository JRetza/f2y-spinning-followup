Figure \ref{fig:comp_masses} compares the $90\%$ credible regions in component-mass space of $5$ chosen simulated signals \citep[cf.][figure 1]{Chatziioannou_2014}\footnote{Due to the difficulty of estimating the narrow and nonlinearly correlated credible regions in m1–m2 space, we illustrate the credible regions in $m_1$--$m_2$ space as the projection of a rectangular region in $\mathcal{M}_\mathrm{c}$--$q$space, bounded above and below by the 1D $5^\mathrm{th}$ and $95^\mathrm{th}$ percentiles, respectively}. If the analyses used a prior distribution for spin matching the simulated population ($\chi < 0.05$), we would expect the masses of the simulated sources to fall within the estimated $90\%$ credible regions for close to $90\%$ of simulations. For the broad spin prior used ($\chi < 1$) however, this will not be the case.  If BNS systems turn out to have appreciable spins, however, it is evident that making strong (and wrong) prior assumptions about NS spin can drastically bias mass estimates.
  
  